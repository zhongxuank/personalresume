%!TEX TS-program = xelatex
%!TEX encoding = UTF-8 Unicode
% Awesome CV LaTeX Template for Cover Letter
%
% This template has been downloaded from:
% https://github.com/posquit0/Awesome-CV
%
% Authors:
% Claud D. Park <posquit0.bj@gmail.com>
% Lars Richter <mail@ayeks.de>
%
% Template license:
% CC BY-SA 4.0 (https://creativecommons.org/licenses/by-sa/4.0/)
%


%-------------------------------------------------------------------------------
% CONFIGURATIONS
%-------------------------------------------------------------------------------
% A4 paper size by default, use 'letterpaper' for US letter
\documentclass[11pt, a4paper]{awesome-cv}

% Configure page margins with geometry
\geometry{left=1.8cm, top=.8cm, right=1.8cm, bottom=1.8cm, footskip=.5cm}

% Specify the location of the included fonts
\fontdir[fonts/]

% Color for highlights
% Awesome Colors: awesome-emerald, awesome-skyblue, awesome-red, awesome-pink, awesome-orange
%                 awesome-nephritis, awesome-concrete, awesome-darknight
\colorlet{awesome}{awesome-skyblue}
% Uncomment if you would like to specify your own color
% \definecolor{awesome}{HTML}{CA63A8}

% Colors for text
% Uncomment if you would like to specify your own color
% \definecolor{darktext}{HTML}{414141}
% \definecolor{text}{HTML}{333333}
% \definecolor{graytext}{HTML}{5D5D5D}
% \definecolor{lighttext}{HTML}{999999}

% Set false if you don't want to highlight section with awesome color
\setbool{acvSectionColorHighlight}{true}

% If you would like to change the social information separator from a pipe (|) to something else
\renewcommand{\acvHeaderSocialSep}{\quad\textbar\quad}


%-------------------------------------------------------------------------------
%	PERSONAL INFORMATION
%	Comment any of the lines below if they are not required
%-------------------------------------------------------------------------------
% Available options: circle|rectangle,edge/noedge,left/right
% \photo[circle,noedge,left]{profile/me_square}
\name{Zhong Xuan}{Khwa}
% \position{Bachelor's Student, Yale-NUS College}
% \address{42-8, Bangbae-ro 15-gil, Seocho-gu, Seoul, 00681, Rep. of KOREA}

\extrainfo{Yale-NUS College, Singapore}
\mobile{(+65) 9696 3373}
\email{zhongxuan@u.yale-nus.edu.sg}
% \homepage{www.posquit0.com}
% \github{posquit0}
\linkedin{zhongxuan95}
% \gitlab{gitlab-id}
% \stackoverflow{SO-id}{SO-name}
% \twitter{@twit}
% \skype{skype-id}
% \reddit{reddit-id}
% \medium{madium-id}
% \googlescholar{googlescholar-id}{name-to-display}
%% \firstname and \lastname will be used
% \googlescholar{googlescholar-id}{}

% \quote{``Be the change that you want to see in the world."}


%-------------------------------------------------------------------------------
%	LETTER INFORMATION
%	All of the below lines must be filled out
%-------------------------------------------------------------------------------
% The company being applied to
\recipient
  {Company Recruitment Team}
  {Google Inc.\\1600 Amphitheatre Parkway\\Mountain View, CA 94043}
% The date on the letter, default is the date of compilation
\letterdate{\today}
% The title of the letter
\lettertitle{Job Application for Software Engineer}
% How the letter is opened
\letteropening{Dear Sir / Madam,}
% How the letter is closed
\letterclosing{Sincerely,}
% Any enclosures with the letter
\letterenclosure[Attached]{Curriculum Vitae}


%-------------------------------------------------------------------------------
\begin{document}

% Print the header with above personal informations
% Give optional argument to change alignment(C: center, L: left, R: right)
\makecvheader[R]

% Print the footer with 3 arguments(<left>, <center>, <right>)
% Leave any of these blank if they are not needed
\makecvfooter
  {\today}
  {Zhong Xuan Khwa~~~·~~~Cover Letter}
  {}

% Print the title with above letter informations
\makelettertitle

%-------------------------------------------------------------------------------
%	LETTER CONTENT
%-------------------------------------------------------------------------------
\begin{cvletter}

% \lettersection{About Me}
  I hope to pursue research in the field of Artificial Intelligence (AI),
  particularly in the intersection between theoretical mathematics and AI.
  During my undergraduate studies in Yale-NUS College, I was privileged in
  receiving a robust, multidisciplinary education, especially in the areas of
  Mathematics, Statistics, and Computer Science. I had been drawn towards topics
  in theoretical Mathematics, such as discrete mathematics, graph theory, and
  modern algebra, while also having a passion for deep learning and AI. Thus, I
  worked towards finding opportunities in the intersection of these two fields.
  I pursued this during my research project with Professor Isabelle Augenstein
  at her CopeNLU lab in Copenhagen, Denmark. The project, among many things,
  involved exploring the effectiveness of Neural Networks in learning hidden
  structures of knowledge graphs. It exposed me to the limited effectiveness of
  current Deep Learning techniques when used on non-conventional mathematical
  structures, like graphs, and indicated a potential for important research to
  be conducted.

  The programme offered by the University of Amsterdam (UvA) is exactly what I’m
  looking for as my next step on this journey due to UvA’s particular focus on
  AI research in a multidisciplinary setting, and the research-oriented nature
  of the Master’s programme itself. Looking into the research done at UvA, I was
  excited to find an impressive collection of theoretical AI papers, such as the
  papers on Graph Convolutional Networks by Thomas Kipf and Max Welling, which I
  had referenced in my research project. I was also particularly impressed by
  UvA’s research on AI’s wider impact on society as well, as it demonstrates
  UvA’s leadership in responsible, ethical AI development.

  Furthermore, the AI Master programme’s strong emphasis on providing a solid
  foundation across the major fields in AI is exactly what I am looking to gain
  during my Master’s programme. The required modules provide a robust coverage
  of the fundamentals of AI, and the flexibility in the electives allows me to
  deep dive into specific interest areas. My previous coursework has allowed me
  to build a strong foundation in Mathematics and Statistics, and the
  theoretical foundations provided in the course would be the perfect bridge
  between my experiences and the research I hope to do.

  Finally, I wish to be nominated for the Amsterdam Excellence Scholarship, as
  it would enable me to afford this program. Coming from a humble background,
  receiving the Global Leader Scholarship from my undergraduate institution
  allowed me to afford the cost of my Bachelors degree. Having attained a GPA of
  4.73 / 5.00, while concurrently undertaking a long list of leadership roles in
  the college, I believe that I have embodied the expectations set upon me as a
  scholarship recipient. Through the Amsterdam Excellence Scholarship, I do hope
  to have the

\end{cvletter}


%-------------------------------------------------------------------------------
% Print the signature and enclosures with above letter informations
\makeletterclosing

\end{document}

%%% Local Variables: 
%%% coding: utf-8
%%% mode: latex
%%% TeX-engine: xetex
%%% TeX-master: t
%%% End: 